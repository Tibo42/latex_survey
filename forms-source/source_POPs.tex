\renewcommand{\name}{POPs}

% Grolnick, Wendy S., Richard M. Ryan, and Edward L. Deci. "Inner resources for school achievement: Motivational mediators of children's perceptions of their parents." Journal of educational psychology 83.4 (1991): 508.

%Scales
% 3, 6 (R), 9, 12 (R), 15 (R), 18  		Implication des ref\'erents -Im
% 1, 2 (R), 5, 8, 11, 14 (R), 17, 19, 21 (R) 	Soutien \`a l'autonomie des ref\'erents -SA
% 4, 7, 10, 13 (R), 16, 20 (R)			Chaleur des ref\'erents -Ch

%%%%%%%%%%%%%%%%%%%%%%%%%%%%%%%%%%
%\newcommand{\RefA}[1][Mes]{#1 enseignants} 
%\newcommand{\RefB}[1][Mes]{#1 parents} 
%\newcounter{Cpt}\setcounter{Cpt}{1}
%\definecolor{my-color-box}{gray}{1}
%\setlength{\fboxsep}{5pt}
\renewcommand{\Qbox}[2][0.85]{\vspace*{-0.5em}\parbox[t]{#1\textwidth}{#2}}
%\newcommand{\setspace}{\hspace*{-1.5em}}
%%%%%%%%%%%%%%%%%%%%%%%%%%%%%%%%%%
\def\stylePOPs{\def\AMCbeginQuestion##1##2{\par\noindent{\bf\theCpt.}\hspace*{0.25em}}}
%%%%%%%%%%%%%%%%%%%%%%%%%%%%%%%%%%
\renewcommand{\LikertSeven}[2]{
\element{\name}{
{\stylePOPs
\begin{question}{#1}
	\stepcounter{Cpt}
  \Qbox{#2}
  \begin{reponseshoriz}[o]
  \begin{small}
  	\fcolorbox{black}{my-color-box}{
		  \textit{Pas vrai}~\hspace{0.6cm}
		  \mauvaise{\setspace}\bareme{-3}
		  \mauvaise{\setspace}\bareme{-2}
		  \mauvaise{\setspace}\bareme{-1}
		  \mauvaise{\setspace}\bareme{0}
		  \mauvaise{\setspace}\bareme{1}
		  \mauvaise{\setspace}\bareme{2}
		  \bonne{\setspace}\bareme{3}
		  \hspace{-0.6cm}~\textit{Vrai}
    }
  \end{small}
  \end{reponseshoriz}
\end{question}
}}}
\renewcommand{\LikertSevenNeg}[2]{
\element{\name}{
{\stylePOPs
\begin{question}{#1}
	\stepcounter{Cpt}
  \Qbox{#2}
  \begin{reponseshoriz}[o]
  \begin{small}
  	\fcolorbox{black}{my-color-box}{
		  \textit{Pas vrai}~\hspace{0.6cm}
		  \bonne{\setspace}\bareme{3}
		  \mauvaise{\setspace}\bareme{2}
		  \mauvaise{\setspace}\bareme{1}
		  \mauvaise{\setspace}\bareme{0}
		  \mauvaise{\setspace}\bareme{-1}
		  \mauvaise{\setspace}\bareme{-2}
		  \mauvaise{\setspace}\bareme{-3}
		  \hspace{-0.6cm}~\textit{Vrai}
    }
  \end{small}
  \end{reponseshoriz}
\end{question}
}}}
%%%%%%%%%%%%%%%%%%%%%%%%%%%%%%%%%%
\renewcommand{\consigne}{
Indiquez dans quelle mesure chacun des \'enonc\'es suivants sont, selon vous, \textbf{vrai} ou \textbf{pas vrai}\\}
\renewcommand{\intro}{\vspace{-8ex}}
%questions:
\newcommand\Rtype[2]{
\LikertSeven{1-#1-SA}{#2~semblent savoir ce que je ressens \`a propos des choses.}
\LikertSevenNeg{2-#1-SA}{#2~essaient de me dire comment g\'erer ma vie.}
\LikertSeven{3-#1-Im}{#2~trouvent le temps de parler avec moi.}
\LikertSeven{4-#1-Ch}{#2~m'acceptent et m'appr\'ecient comme je suis.}
\LikertSeven{5-#1-SA}{#2~me permettent, autant que possible, de choisir quoi faire.}
\LikertSevenNeg{6-#1-Im}{#2~ne semblent pas penser \`a moi souvent.}
\LikertSeven{7-#1-Ch}{#2~transmettent clairement leur affection pour moi.}
\LikertSeven{8-#1-SA}{#2~\'ecoutent mon opinion ou mon point de vue quand j'ai un probl\`eme.}
\LikertSeven{9-#1-Im}{#2~passent beaucoup de temps avec moi.}
\LikertSeven{10-#1-Ch}{#2~me font me sentir tr\`es sp\'ecial.}
\LikertSeven{11-#1-SA}{#2~me permettent de d\'ecider des choses pour moi-m\^eme.}
\LikertSevenNeg{12-#1-Im}{#2~semblent souvent trop occup\'es pour s'occuper de moi.}
\LikertSevenNeg{13-#1-Ch}{#2~sont souvent d\'esapprobateurs et ne m'acceptent pas.}
\LikertSevenNeg{14-#1-SA}{#2~insistent pour que je fasse les choses \`a leur fa\c{c}on.}
\LikertSevenNeg{15-#1-Im}{#2~ne sont pas tr\`es impliqu\'es dans mes pr\'eoccupations.}
\LikertSeven{16-#1-Ch}{#2~sont g\'en\'eralement heureux de me voir.}
\LikertSeven{17-#1-SA}{#2~sont g\'en\'eralement pr\^ets \`a consid\'erer les choses de mon point de vue.}
\LikertSeven{18-#1-Im}{#2~mettent du temps et de l'\'energie pour m'aider.}
\LikertSeven{19-#1-SA}{#2~m'aident \`a choisir ma propre direction.}
\LikertSevenNeg{20-#1-Ch}{#2~semblent \^etre beaucoup d\'e\c{c}us par moi.}
\LikertSevenNeg{21-#1-SA}{#2~ne sont pas tr\`es sensibles \`a beaucoup de mes besoins.}
}
%%%%%%%%%%%%%%%%%%%%%%%%%%%%%%%%%%	SET question
\Rtype{En}{\RefA}
\copygroup{\name}{\name-enseignants}
\cleargroup{\name}
\Rtype{Pa}{\RefB}
\copygroup{\name}{\name-parents}
\cleargroup{\name}
%%%%%%%%%%%%%%%%%%%%%%%%%%%%%%%%%%
