\renewcommand{\name}{SPP}
%Self-Perception Profile (SPP) Harter, S. (1982).

%Harter, Susan. "The perceived competence scale for children." Child development (1982): 87-97.

% 6 sub-scale:

% 1b, 7b, 13b , 19, 25b \'ecole (perception des comp\'etences) -Eco
% 2b, 8, 14b, 20b, 26 Social -Socio
% 3, 9b, 15, 21, 27b Physique -Phy
% 4, 10b, 16b, 22b, 28 Appearance -App
% 5b, 11, 17, 23,29 Conduite -Cond
% 6b, 12, 18, 24, 30b Valeur de soi (estime de soi globale) -Val

%https://journals.openedition.org/osp/1130?file=1
%https://sci-hub.tw/10.2307/1129640
%http://semaphore.uqar.ca/93/1/Annie_D%27Amours_juillet2009.pdf
%https://journals.openedition.org/osp/1118

%%%%%%%%%%%%%%%%%%%%%%%%%%%%%%%%%%
%\newcommand{\cible}[1]{#1 jeunes} 
%\newcounter{Cpt}\setcounter{Cpt}{1}
%\definecolor{my-color-box}{gray}{1}
%\setlength{\fboxsep}{7pt}
%\newcommand{\Qbox}[2][0.85]{\par\noindent{\bf#1.} #2\hspace*{0.4em}}
%\newcommand{\setspace}{\hspace*{-1.5em}}
%%%%%%%%%%%%%%%%%%%%%%%%%%%%%%%%%%
\def\styleSPP{\def\AMCbeginQuestion##1##2{}}
%%%%%%%%%%%%%%%%%%%%%%%%%%%%%%%%%%
\renewcommand{\ProfilA}[4][Certains]{
\element{\name}{
{\styleSPP
\begin{question}{#2}\AMCBoxedAnswers
	\stepcounter{Cpt}
  \begin{reponses}[o]
		\boxput*(0,1){\colorbox{white}{\cible{#1}}}{
  	\fcolorbox{black}{white}{
  		\fcolorbox{black}{my-color-box}{\parbox{2.3cm}{
  				\centering\small\textit{Me ressemble}
					\mauvaise{Un peu}\bareme{1}
					\bonne{Beaucoup}\bareme{2}
			}}
    	\fcolorbox{white}{white}{\parbox{3.2cm}{\centering #3}
    	}
			\fcolorbox{black}{my-color-box}{\parbox{1.1cm}{\centering Mais\\d'autres}
			}
    	\fcolorbox{white}{white}{\parbox{3.2cm}{\centering #4}
    	}
    	\fcolorbox{black}{my-color-box}{\parbox{2.3cm}{
					\centering\small\textit{Me ressemble}
					\mauvaise{Un peu}\bareme{-1}
					\mauvaise{Beaucoup}\bareme{-2}
			}}
    }}
  \end{reponses}
\end{question}
}}}
\renewcommand{\ProfilB}[4][Certains]{
\element{\name}{
{\styleSPP
\begin{question}{#2}\AMCBoxedAnswers
	\stepcounter{Cpt}
  \begin{reponses}[o]
		\boxput*(0,1){\colorbox{white}{\cible{#1}}}{
  	\fcolorbox{black}{white}{
  		\fcolorbox{black}{my-color-box}{\parbox{2.3cm}{
  				\centering\small\textit{Me ressemble}
					\mauvaise{Un peu}\bareme{-1}
					\mauvaise{Beaucoup}\bareme{-2}
			}}
    	\fcolorbox{white}{white}{\parbox{3.2cm}{\centering #3}
    	}
			\fcolorbox{black}{my-color-box}{\parbox{1.1cm}{\centering Mais\\d'autres}
			}
    	\fcolorbox{white}{white}{\parbox{3.2cm}{\centering #4}
    	}
    	\fcolorbox{black}{my-color-box}{\parbox{2.3cm}{
					\centering\small\textit{Me ressemble}
					\mauvaise{Un peu}\bareme{1}
					\bonne{Beaucoup}\bareme{2}
			}}
    }}
  \end{reponses}
\end{question}
}}}
%%%%%%%%%%%%%%%%%%%%%%%%%%%%%%%%%%
\renewcommand{\consigne}{

<< \`A qui je ressemble >>\\
\vspace{2ex}
Ce qui nous int\'eresse, c'est de savoir ce qu'est chacun de vous, quel type de personne vous \^etes.\\C'est une enqu\^ete, pas un test. Il n'y a ni bonnes ni mauvaises r\'eponses. Puisque \cible{les} sont tr\`es diff\'erents les uns des autres, chacun de vous marquera quelque chose de diff\'erent.\\
\vspace{2ex}
Pour chaque \'enonc\'e, deux profils d'individus sont pr\'esent\'es:\\ \textit{\cible{Certains}} \dots profils A \dots \textit{mais d'autres} \dots profils B \dots\\
\vspace{2ex}
Apr\`es avoir selectionn\'e un seul de ces deux profils, indiquez si celui-ci vous ressemble\\ \textit{Un peu} ou \textit{Beaucoup}.\\
\vspace{2ex}
Pour chacun des \'enonc\'e, vous ne s\'electionnez qu'une seule case.\\Quelquefois la case que vous selectionnerez sera d'un c\^ot\'e de la page, d'autres fois elle sera de l'autre c\^ot\'e, mais vous ne devez cocher qu'une seule case pour chaque phrase.\\Vous ne cochez pas des deux c\^ot\'es, seulement du c\^ot\'e du profil qui vous ressemble le plus.}
\renewcommand{\intro}{\vspace{-5ex}}
%questions:
\ProfilA{1A-Eco}{ont l'impression de bien travailler \`a l'\'ecole}{se demandent s' ils travaillent suffisamment}
\ProfilB{2B-Socio}{trouvent difficile de se faire des amis}{trouvent que c'est facile}
\ProfilA{3A-Phy}{se sentent dou\'es pour toutes sortes de sports}{ne se sentent pas tellement dou\'es pour le sport}
\ProfilA{4A-App}{sont satisfaits de leur taille et de leur poids}{aimeraient bien que leur taille et leur poids soient diff\'erents}
\ProfilB[Des]{5B-Cond}{sont souvent peu satisfaits de leur conduite}{sont plut\^ot satisfaits de leur conduite}
\ProfilB[Il y a des]{6B-Val}{qui ne sont pas satisfaits de leur vie}{qui sont satisfaits de leur vie}
\ProfilB{7B-Eco}{travaillent lentement \`a l'\'ecole}{font leur travail rapidement}
\ProfilA{8A-Socio}{ont un tas de copains}{n'ont pas tellement de copains}
\ProfilB{9B-Phy}{voudraient bien \^etre meilleurs en sport}{se sentent assez bons comme \c{c}a}
\ProfilB[Des]{10B-App}{aimeraient que leur corps soit un peu diff\'erent}{aiment bien le corps qu'ils ont}
\ProfilA{11A-Cond}{font toujours les choses bien comme il faut}{la plupart du temps, ne font pas les choses comme il faut}
\ProfilA{12A-Val}{sont, la plupart du temps, contents d'eux-m\^emes}{ne sont souvent pas contents d'eux-m\^emes}
\ProfilB[Des]{13B-Eco}{oublient souvent ce qu 'ils ont appris}{peuvent se rappeler facilement les choses}
\ProfilB[Il y a des]{14B-Socio}{qui ont de la peine \`a se faire aimer}{qui savent bien se faire aimer}
\ProfilA{15A-Phy}{pensent qu 'ils arriveraient \`a faire n'importe quel exercice de gym}{craignent un peu de ne pas r\'eussir les nouveaux exercices}
\ProfilB[Des]{16B-App}{voudraient bien avoir une apparence un peu diff\'erente}{aiment bien leur apparence}
\ProfilA{17A-Cond}{font d' habitude les choses comme on leur demande}{ne font pas toujours comme on leur demande}
\ProfilA{18A-Val}{aiment bien le genre de personne qu'ils sont}{aimeraient souvent \^etre quelqu ' un d'autre}
\ProfilA[Des]{19A-Eco}{font tr\`es bien leur travail en classe}{ne font pas tr\`es bien leur travail en classe}
\ProfilB{20B-Socio}{voudraient qu ' il y ait plus de personnes qui les aiment}{pensent que la plupart des personnes les aiment bien}
\ProfilA{21A-Phy}{trouvent qu ' ils sont meilleurs en sport que leurs copains}{se sentent moins bons qu'eux}
\ProfilB[Des]{22B-App}{voudraient bien que leur visage ou leurs cheveux soient un peu diff\'erents}{aiment bien leur visage et leurs cheveux comme ils sont}
\ProfilB{23B-Cond}{font des choses en sachant qu'ils ne devraient pas les faire}{ne font pratiquement jamais des choses qu'ils ne devraient pas faire}
\ProfilA[Des]{24A-Val}{sont tr\`es contents d'\^etre ce qu'ils sont}{voudraient bien \^etre diff\'erents}
\ProfilB[A l'\'ecole, certains]{25B-Eco}{ont de la peine \`a imaginer des r\'eponses aux questions}{parviennent presque toujours \`a imaginer}
\ProfilA{26A-Socio}{sont bien appr\'eci\'es par leurs copains}{ne sont pas tellement appr\'eci\'es}
\ProfilB[Aux jeux ou aux sports, certains]{27B-Phy}{pr\'ef\`erent regarder plut\^ot que jouer}{pr\'ef\`erent jouer plut\^ot que regarder}
\ProfilA{28A-App}{pensent qu'ils pr\'esentent bien}{pensent qu'ils ne pr\'esentent pas tellement bien.}
\ProfilA{29A-Cond}{sont d'habitude tr\`es agr\'eables avec les autres}{pensent qu'ils pourraient parfois \^etre plus agr\'eables}
\ProfilB{30B-Val}{sont souvent m\'econtents de ce qu'ils font}{sont plut\^ot contents de ce qu ' ils font}
%%%%%%%%%%%%%%%%%%%%%%%%%%%%%%%%%%
